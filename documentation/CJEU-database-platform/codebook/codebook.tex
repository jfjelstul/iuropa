%--------------------------------------------------%
% generated by the codebookr R package
% created by Joshua C. Fjelstul, Ph.D.
%--------------------------------------------------%

\documentclass[10pt]{article}

%--------------------------------------------------%
% packages
%--------------------------------------------------%

% page layout
\usepackage{geometry}

% fonts
\usepackage[english]{babel}
\usepackage{underscore}
\usepackage{anyfontsize}
\usepackage[utf8]{inputenc}
\usepackage[T1]{fontenc}
\usepackage{fontspec}

% graphics and tables
\usepackage{graphicx} % add figures
\usepackage{xcolor} % change font color
\usepackage{tikz} % add graphics

% paragraph spacing
\usepackage{setspace}

% hyperlinks
\usepackage{url}

% table of contents
\usepackage{tocloft}

% test alignment
\usepackage{ragged2e}

% multi-page tables
\usepackage{longtable}

% custom lists
\usepackage{enumitem}

% insert content on every page
\usepackage{atbegshi} 

% code formatting
\usepackage{tcolorbox}

%--------------------------------------------------%
% colors
%--------------------------------------------------%

% define colors
\definecolor{themecolor}{HTML}{5D92E0}
\definecolor{background}{HTML}{EEF6FD}

% format hyperlinks
\usepackage[colorlinks=true,linkcolor=themecolor,citecolor=themecolor,urlcolor=themecolor,breaklinks=true]{hyperref}

%--------------------------------------------------%
% formatting
%--------------------------------------------------%

% configure main font
\setmainfont[Ligatures=TeX,BoldFont={Roboto Medium}]{Roboto Light}
\setmonofont[Ligatures=TeX]{Roboto Mono-Light}

% set page margins
\geometry{top = 1.5in, bottom = 1.5in, left = 1.5in, right = 1.5in}

% set paper size
\geometry{letterpaper}

% format table of contents
\renewcommand{\cftsecdotsep}{10}
\renewcommand{\cftsecleader}{\cftdotfill{\cftdotsep}}
\renewcommand{\cftsecfont}{{\small\color{black!75}\bfseries}}
\renewcommand{\cftsecpagefont}{{\small\color{black!75}\normalfont}}

% adjust spacing
\usepackage{parskip}
\parskip=10pt
\renewcommand{\baselinestretch}{1.4}

% hyphen formatting
\hyphenpenalty = 10000
\exhyphenpenalty = 10000

% prevent widow and orphan lines
\widowpenalty10000
\clubpenalty10000

%--------------------------------------------------%
% page elements
%--------------------------------------------------%

% a command to make a code box
\newtcbox{\codebox}{nobeforeafter,tcbox raise base,colback=black!5,colframe=white,coltext=black!75,boxrule=0pt,arc=3pt,boxsep=0pt,
left=4pt,right=4pt,top=3pt,bottom=3pt}

% a command to make a chip
\newtcbox{\chip}{nobeforeafter,tcbox raise base,colback=black!5,colframe=white,coltext=black!75,boxrule=0pt,arc=11pt,boxsep=0pt,
left=10pt,right=10pt,top=8pt,bottom=8pt}

% command to format code
\newcommand{\code}[1]{\codebox{{\footnotesize\texttt{#1}}}}

% command to highlight text
\newcommand{\highlight}[1]{{\color{themecolor} \textbf{#1}}}

% command to create a divider
\newcommand{\dividerline}{{\color{gray!10} \rule[4pt] {\textwidth}{3pt}}}

% command to add a cover
\newcommand{\cover}[4]{
\begin{tikzpicture}[remember picture,overlay, shift={(current page.south west)}]
\fill[themecolor] (0, 5.5in) rectangle ++ (8.5in, 5.5in); % header bar
\fill[black!5] (0, 4in) rectangle ++ (8.5in, 1.5in); % middle bar
\fill[white] (0, 0in) rectangle ++ (8.5in, 4in); % footer bar
\node[anchor=west] at (1.5in, 6.25in) {\color{white} \fontsize{60}{60}\selectfont \begin{minipage}{5.5in} \textbf{Codebook} \fontsize{15}{15}\selectfont \hspace{5pt} v #2 \end{minipage}};
\node[anchor=west, align=left] at (1.5in, 4.75in) {\begin{minipage}{5.5in} \color{black!40} \fontsize{#4}{#4} \selectfont #1 \end{minipage}};
\node[anchor=west, align=left, minimum height=2in] at (1.5in, 2.55in) {\begin{minipage}[t][2in]{5.5in} \color{black!40} \fontsize{10}{10} \selectfont #3 \end{minipage}};
\end{tikzpicture}
}

% command to add a header page
\newcommand{\headerpage}[4]{
	\newpage
	\begin{tikzpicture}[remember picture,overlay, shift={(current page.south west)}]
		\fill[themecolor] (0, 9in) rectangle ++ (8.5in, 2in); % header line 1
		\fill[black!5] (0, 8in) rectangle ++ (8.5in, 1in); % header line 2
		\node[anchor = west] at (1.5in, 9.6in) {\color{white} \fontsize{#3}{#3}\selectfont \textbf{#1}}; % heading
		\node[anchor = west] at (1.5in, 8.5in) {\color{black!40} \fontsize{#4}{#4}\selectfont #2}; % heading
	\end{tikzpicture}
	\phantomsection
	\addcontentsline{toc}{section}{#1}
	\vspace{1.5in}
}

% command to layout page
\newcommand\pagelayout{
	\begin{tikzpicture}[remember picture,overlay, shift={(current page.south west)}]
		% \fill[themecolor] (0, 10.75in) rectangle ++ (8.5in, 0.25in); % header
		\fill[black!5] (0, 0) rectangle ++ (8.5in, 0.5in); % footer
		\draw (0.25in, 0.25in) node[anchor = west] {\fontsize{9}{9}\selectfont \color{black!40} The CJEU Database Platform Codebook \hspace{5pt} | \hspace{5pt} The IUROPA Project}; % footer content
		\draw (8.25in, 0.25in) node[anchor = east] {\fontsize{9}{9}\selectfont \color{black!40} \thepage}; % page number
	\end{tikzpicture}
}

% add page layout 
\AtBeginShipout{
	\AtBeginShipoutUpperLeft{\pagelayout}
}

% command to add a subheading
\newcommand{\subheading}[1]{
\vspace{24pt}
{\color{themecolor} \fontsize{14}{14}\selectfont \textbf{#1}}
\vspace{6pt}
\dividerline
\vspace{-20pt}
}

%--------------------------------------------------%
% start document
%--------------------------------------------------%

\begin{document}

\clearpage
\pagestyle{empty}

\color{black!75}

\small

\begin{flushleft}

%--------------------------------------------------%
% cover
%--------------------------------------------------%

\cover{The CJEU Database Platform}{0.1}{Stein Arne Brekke\\[0.75em]Joshua Fjelstul\\[0.75em]Silje Synnøve Lyder Hermansen\\[0.75em]Daniel Naurin}{16}

\newpage

%--------------------------------------------------%
% table of contents
%--------------------------------------------------%

% reset page counter
\setcounter{page}{1}

% format the table of contents header
% \renewcommand\contentsname{{\color{themecolor} \fontsize{14}{14}\selectfont Datasets}}
\renewcommand\contentsname{\subheading{Datasets} \vspace{0pt}}

% add the table of contents
\tableofcontents

% remove page number from table of contents pages
\addtocontents{toc}{\protect\thispagestyle{empty}}

\newpage

%--------------------------------------------------%
% content
%--------------------------------------------------%


%--------------------------------------------------%
% dataset
%--------------------------------------------------%

\headerpage{cases}{CJEU cases}{35}{12}

\subheading{Description}

This dataset includes information on the universe of CJEU cases. There is one observation per case (before joins). The dataset indicates the title of the case, how cases are joined together into proceedings, the date of the hearing (if applicable), the dates and ECLI numbers of documents in the case (judgments, orders, AG opinions), the EUR-Lex subject codes for the case, and more.

\subheading{Variables}

\begin{description}[labelwidth=130pt, leftmargin=\dimexpr\labelwidth+\labelsep\relax, font=\normalfont, itemsep=10pt]
\item[\code{key\_id}] \code{numeric}\hspace{5pt}An ID number that uniquely identifies each observation in the dataset. 
\item[\code{court\_id}] \code{numeric}\hspace{5pt}An ID number that uniquely identifies each court. Coded \code{1} for \code{Court of Justice}, \code{2} for \code{General Court}, and \code{3} for \code{Civil Service Tribunal}.
\item[\code{court}] \code{string}\hspace{5pt}The name of the court. Either \code{Court of Justice}, \code{General Court}, or \code{Civil Service Tribunal}.
\item[\code{iuropa\_case\_id}] \code{string}\hspace{5pt}An ID number that uniquely identifies each case in the format \code{CJEU:X:\#\#\#\#:\#\#\#\#}, where \code{X} indicates the court (\code{C} for the Court of Justice, \code{T} for the General Court, and \code{F} for the Civil Service Tribunal), the first set of digits is the year the proceeding was opened, and the second set of digits is the number of the proceeding with leading zeros.
\item[\code{case}] \code{string}\hspace{5pt}The case number in the format \code{C-\#\#\#/\#\#} for Court of Justice proceedings, \code{T-\#\#\#/\#\#} for General Court proceedings, and \code{F-\#\#\#/\#\#} for Civil Service Tribunal proceedings. The first set of digits is the number of the proceeding and the second set of digits is the last two digits of the year the proceeding was opened. When multiple cases are joined, the proceeding number for the joined cases is the lowest case number.
\item[\code{case\_year}] \code{numeric}\hspace{5pt}The year of the case, used in \code{proceeding} and \code{proceeding\_id}.
\item[\code{case\_number}] \code{numeric}\hspace{5pt}The number of the case, used in \code{case} and \code{case\_id}.
\item[\code{case\_date}] \code{date}\hspace{5pt}The date the case was opened in the format \code{YYYY-MM-DD}.
\item[\code{iuropa\_proceeding\_id}] \code{string}\hspace{5pt}An ID number that uniquely identifies each proceeding in the format \code{CJEU:X:\#\#\#\#:\#\#\#\#}, where \code{X} indicates the court (\code{C} for the Court of Justice, \code{T} for the General Court, and \code{F} for the Civil Service Tribunal), the first set of digits is the year the proceeding was opened, and the second set of digits is the number of the proceeding with leading zeros.
\item[\code{proceeding}] \code{string}\hspace{5pt}The proceeding number in the format \code{C-\#\#\#/\#\#} for Court of Justice proceedings, \code{T-\#\#\#/\#\#} for General Court proceedings, and \code{F-\#\#\#/\#\#} for Civil Service Tribunal proceedings. The first set of digits is the number of the proceeding and the second set of digits is the last two digits of the year the proceeding was opened. When multiple cases are joined, the proceeding number for the joined cases is the lowest case number.
\item[\code{proceeding\_year}] \code{numeric}\hspace{5pt}The year of the proceeding, used in \code{proceeding} and \code{proceeding\_id}. The year of the earliest case associated with the proceeding.
\item[\code{proceeding\_number}] \code{numeric}\hspace{5pt}The number of the proceeding, used in \code{proceeding} and \code{proceeding\_id}. The proceeding number is the case number of the earliest case associated with the proceeding.
\item[\code{case\_title}] \code{string}\hspace{5pt}The title of the case.
\item[\code{joined\_cases}] \code{string}\hspace{5pt}A list of all case numbers associated with the proceeding. If only one case number is listed, the case number (see \code{case}) and the proceeding number (see \code{proceeding}) will be the same. If there are multiple cases that were joined together, the proceeding number and the case number will only be the same for the earliest of the joined cases.
\item[\code{origin}] \code{string}\hspace{5pt}The origin of the case. Usually a member state. If there are multiple origins, they are listed, separated by a comma. Coded \code{NA} if not applicable or if missing.
\item[\code{date\_hearing}] \code{date}\hspace{5pt}The date of the hearing in the format \code{YYYY-MM-DD}, if applicable. Coded \code{NA} if not applicable or if missing.
\item[\code{date\_order}] \code{string}\hspace{5pt}The dates of any orders associated with the case in the format \code{YYYY-MM-DD}, separated by a semicolon.
\item[\code{date\_ag\_opinion}] \code{string}\hspace{5pt}The dates of any AG opinions associated with the case in the format \code{YYYY-MM-DD}, separated by a semicolon.
\item[\code{date\_judgment}] \code{string}\hspace{5pt}The dates of any judgments associated with the case in the format \code{YYYY-MM-DD}, separated by a semicolon.
\item[\code{judgement}] \code{string}\hspace{5pt}The ECLI numbers for any judgments in the case, if applicable. If there are multiple judgments, their ECLI numbers are listed, separated by a comma. Coded \code{NA} if not applicable or if missing.
\item[\code{order}] \code{string}\hspace{5pt}The ECLI numbers for any orders associated with the case, if applicable. If there are multiple orders, their ECLI numbers are listed, separated by a comma. Coded \code{NA} if not applicable or if missing.
\item[\code{ag\_opinion}] \code{string}\hspace{5pt}The ECLI number for any AG opinions associated with the case, if applicable. If there are multiple AG opinions, their ECLI numbers are listed, separated by a comma. Coded \code{NA} if not applicable or if missing.
\item[\code{advocate\_general\_id}] \code{numeric}\hspace{5pt}An ID number that uniquely identifies the advocate general assigned to the case, if applicable. Coded \code{NA} if not applicable or if missing.
\item[\code{advocate\_general}] \code{string}\hspace{5pt}The name of the advocate general assigned to the case, if applicable. Coded \code{NA} if not applicable or if missing.
\item[\code{subject\_matter\_category}] \code{string}\hspace{5pt}The subject matter tags assigned to the case, separated by a semicolon. Coded \code{NA} if missing.
\item[\code{subject\_matter\_subcategory}] \code{string}\hspace{5pt}The subject matter subcategory tags assigned to the case, separated by a semicolon. Coded \code{NA} if missing.
\end{description}
%--------------------------------------------------%
% dataset
%--------------------------------------------------%

\headerpage{parties}{Parties in CJEU cases}{35}{12}

\subheading{Description}

This dataset includes information about the applicant and defendant in each case. There is one observation per party per case. The dataset indicates the actor's name as recorded by the Court, the actor's role in the case (as applicant or defendant), and the type of the actor (member state, EU institution, company, individual etc.).

\subheading{Variables}

\begin{description}[labelwidth=130pt, leftmargin=\dimexpr\labelwidth+\labelsep\relax, font=\normalfont, itemsep=10pt]
\item[\code{key\_id}] \code{numeric}\hspace{5pt}An ID number that uniquely identifies each observation in the dataset. 
\item[\code{court\_id}] \code{numeric}\hspace{5pt}An ID number that uniquely identifies each court. Coded \code{1} for \code{Court of Justice}, \code{2} for \code{General Court}, and \code{3} for \code{Civil Service Tribunal}.
\item[\code{court}] \code{string}\hspace{5pt}The name of the court. Either \code{Court of Justice}, \code{General Court}, or \code{Civil Service Tribunal}.
\item[\code{iuropa\_case\_id}] \code{string}\hspace{5pt}An ID number that uniquely identifies each case in the format \code{CJEU:X:\#\#\#\#:\#\#\#\#}, where \code{X} indicates the court (\code{C} for the Court of Justice, \code{T} for the General Court, and \code{F} for the Civil Service Tribunal), the first set of digits is the year the proceeding was opened, and the second set of digits is the number of the proceeding with leading zeros.
\item[\code{case}] \code{string}\hspace{5pt}The case number in the format \code{C-\#\#\#/\#\#} for Court of Justice proceedings, \code{T-\#\#\#/\#\#} for General Court proceedings, and \code{F-\#\#\#/\#\#} for Civil Service Tribunal proceedings. The first set of digits is the number of the proceeding and the second set of digits is the last two digits of the year the proceeding was opened. When multiple cases are joined, the proceeding number for the joined cases is the lowest case number.
\item[\code{case\_year}] \code{numeric}\hspace{5pt}The year of the case, used in \code{proceeding} and \code{proceeding\_id}.
\item[\code{case\_number}] \code{numeric}\hspace{5pt}The number of the case, used in \code{case} and \code{case\_id}.
\item[\code{case\_date}] \code{date}\hspace{5pt}The date the case was opened in the format \code{YYYY-MM-DD}.
\item[\code{iuropa\_proceeding\_id}] \code{string}\hspace{5pt}An ID number that uniquely identifies each proceeding in the format \code{CJEU:X:\#\#\#\#:\#\#\#\#}, where \code{X} indicates the court (\code{C} for the Court of Justice, \code{T} for the General Court, and \code{F} for the Civil Service Tribunal), the first set of digits is the year the proceeding was opened, and the second set of digits is the number of the proceeding with leading zeros.
\item[\code{proceeding}] \code{string}\hspace{5pt}The proceeding number in the format \code{C-\#\#\#/\#\#} for Court of Justice proceedings, \code{T-\#\#\#/\#\#} for General Court proceedings, and \code{F-\#\#\#/\#\#} for Civil Service Tribunal proceedings. The first set of digits is the number of the proceeding and the second set of digits is the last two digits of the year the proceeding was opened. When multiple cases are joined, the proceeding number for the joined cases is the lowest case number.
\item[\code{proceeding\_year}] \code{numeric}\hspace{5pt}The year of the proceeding, used in \code{proceeding} and \code{proceeding\_id}. The year of the earliest case associated with the proceeded was opened.
\item[\code{proceeding\_number}] \code{numeric}\hspace{5pt}The number of the proceeding, used in \code{proceeding} and \code{proceeding\_id}. The number is assigned based on the order in which the earliest case associated with the proceeding was opened. Case numbers reset at the start of each calendar year.
\item[\code{actor\_role\_id}] \code{numeric}\hspace{5pt}An ID number that uniquely identifies each party role (see \code{actor\_role}). Coded \code{1} for \code{Applicant} and \code{2} for \code{Defendant}.
\item[\code{actor\_role}] \code{string}\hspace{5pt}The role of the party. Either \code{Applicant} or \code{Defendant}.
\item[\code{actor}] \code{string}\hspace{5pt}The name of the party. Coded \code{NA} if not applicable.
\item[\code{actor\_type\_id}] \code{numeric}\hspace{5pt}An ID variable that uniquely identifies the type of the party. Coded \code{1} for \code{EU institution}, \code{2} for \code{Employee of EU institution}, \code{3} for \code{State}, \code{4} for \code{Region}, \code{5} for \code{State institution}, \code{6} for \code{Employee of state institution}, \code{7} for \code{Company}, and \code{8} for \code{Individual}.
\item[\code{actor\_type}] \code{string}\hspace{5pt}The type of the party. Possible values include: \code{EU institution}, \code{Employee of EU institution}, \code{State}, \code{Region}, \code{State institution}, \code{Employee of state institution}, \code{Company}, and \code{Individual}.
\end{description}
%--------------------------------------------------%
% dataset
%--------------------------------------------------%

\headerpage{decisions}{CJEU decisions}{35}{12}

\subheading{Description}

This dataset includes information on the universe of CJEU decisions, including judgments, orders, opinions of the Court, advocate-general opinions, and more. There is one observation per decision. The dataset includes the ECLI and CELEX numbers for each document, which allows you to look up the case in InfoCuria or EUR-Lex. It also indicates whether the proceeding is a direct action or a reference for a preliminary ruling.

\subheading{Variables}

\begin{description}[labelwidth=130pt, leftmargin=\dimexpr\labelwidth+\labelsep\relax, font=\normalfont, itemsep=10pt]
\item[\code{key\_id}] \code{numeric}\hspace{5pt}An ID number that uniquely identifies each observation in the dataset. 
\item[\code{court\_id}] \code{numeric}\hspace{5pt}An ID number that uniquely identifies each court. Coded \code{1} for \code{Court of Justice}, \code{2} for \code{General Court}, and \code{3} for \code{Civil Service Tribunal}.
\item[\code{court}] \code{string}\hspace{5pt}The name of the court. Either \code{Court of Justice}, \code{General Court}, or \code{Civil Service Tribunal}.
\item[\code{iuropa\_proceeding\_id}] \code{string}\hspace{5pt}An ID number that uniquely identifies each proceeding in the format \code{CJEU:X:\#\#\#\#:\#\#\#\#}, where \code{X} indicates the court (\code{C} for the Court of Justice, \code{T} for the General Court, and \code{F} for the Civil Service Tribunal), the first set of digits is the year the proceeding was opened, and the second set of digits is the number of the proceeding with leading zeros.
\item[\code{proceeding}] \code{string}\hspace{5pt}The proceeding number in the format \code{C-\#\#\#/\#\#} for Court of Justice proceedings, \code{T-\#\#\#/\#\#} for General Court proceedings, and \code{F-\#\#\#/\#\#} for Civil Service Tribunal proceedings. The first set of digits is the number of the proceeding and the second set of digits is the last two digits of the year the proceeding was opened. When multiple cases are joined, the proceeding number for the joined cases is the lowest case number.
\item[\code{proceeding\_year}] \code{numeric}\hspace{5pt}The year of the proceeding, used in \code{proceeding} and \code{proceeding\_id}. The year of the earliest case associated with the proceeded was opened.
\item[\code{proceeding\_number}] \code{numeric}\hspace{5pt}The number of the proceeding, used in \code{proceeding} and \code{proceeding\_id}. The number is assigned based on the order in which the earliest case associated with the proceeding was opened. Case numbers reset at the start of each calendar year.
\item[\code{proceeding\_date}] \code{date}\hspace{5pt}The date the proceeding was opened in the format \code{YYYY-MM-DD}. If multiple cases were joined together, this is the date for the earliest of the joined cases.
\item[\code{iuropa\_decision\_id}] \code{string}\hspace{5pt}An ID number that uniquely identifies each decision in the format format \code{CJEU:X:\#\#\#\#:\#\#\#\#:Y:\#\#\#\#\#\#\#\#}, where \code{X} is a letter that indicates the court in question (\code{C} for the Court of Justice, \code{T} for the General Court, and \code{F} for the Civil Service Tribunal, consistent with the letters used in ECLI numbers), the first 4-digit number is the year of the case, the second 4-digit number is the case number (preceded by zeros), \code{Y} is a letter that indicates the type of the document, and the last 8-digit number is the date of the document (in the format \code{YYYYMMDD}). Possible values for \code{Y} are based on the document type indicators in CELEX numbers: \code{J} for judgments, \code{O} for orders, \code{V} for opinions of the Court, \code{D} for decisions, \code{X} for rulings, \code{S} for seizure orders, \code{T} for third-party proceedings, \code{C} for AG opinions, and \code{P} for AG views.
\item[\code{ecli}] \code{string}\hspace{5pt}The ECLI number of the decision. ECLI numbers are assigned by the Court. ECLI numbers have the format \code{ECLI:EU:X:\#\#\#\#:\#\#\#}, where \code{X} is a letter that indicates the court that issued the decision (\code{C} for the Court of Justice, \code{T} for the General Court, and \code{F} for the Civil Service Tribunal), the first set of digits is the year the document was published, and the second set of digits is the number of the decision within the year. Note that ECLI numbers are used for documents besides decisions, so they may not be consecutive for decisions. Note also that the year used in ECLI numbers (the year of publication) is often different from the date in CELEX numbers (the year of the earliest case associated with the proceeding) for the same document.
\item[\code{celex}] \code{string}\hspace{5pt}The CELEX number of the document if the document appears in EUR-Lex. CELEX numbers for CJEU documents have the format \code{6\#\#\#\#XX\#\#\#\#}, where the first set of digits is the year of the proceeding, \code{XX} is a two-letter code that indicates the type of the document, and the last set of digits is the number of the proceeding. If there are multiple CELEX numbers for the same type of document for the same proceeding, subsequent documents will include a suffix in the format \code{(\#\#)}, where the digits indicate the number of the document.
\item[\code{decision\_date}] \code{date}\hspace{5pt}The date that the decision was delivered in the format \code{YYYY-MM-DD}. Coded \code{NA} if missing.
\item[\code{decision\_type\_id}] \code{numeric}\hspace{5pt}An ID number that uniquely identifies each type of decision (see \code{decision\_type}). Coded \code{1} for \code{Judgment}, \code{2} for \code{Order}, \code{3} for \code{Opinion}, \code{4} for \code{Decision}, \code{5} for \code{Ruling}, and \code{6} for \code{AG opinion}.
\item[\code{decision\_type}] \code{string}\hspace{5pt}The type of the decision. Coded \code{NA} if missing. Possible values include: \code{Judgment}, \code{Order}, \code{Opinion}, \code{Decision}, \code{Ruling}, and \code{AG opinion}.
\item[\code{origin}] \code{string}\hspace{5pt}The origin of the case. Usually a member state. If there are multiple origins, they are listed, separated by a comma. Coded \code{NA} if not applicable or if missing.
\item[\code{language\_authentic}] \code{string}\hspace{5pt}The authentic language of the decision. If there are multiple authentic languages, they are listed, separated by a semicolon. Coded \code{NA} if missing.
\item[\code{direct\_action}] \code{dummy}\hspace{5pt}A dummy variable indicating whether the decision involves a direct action.
\item[\code{preliminary\_ruling}] \code{dummy}\hspace{5pt}A dummy variable indicating whether the case involves a reference for a preliminary ruling.
\item[\code{urgent\_procedure}] \code{dummy}\hspace{5pt}A dummy variable indicating whether the Court used the urgent procedure for references for preliminary rulings.
\item[\code{procedure}] \code{string}\hspace{5pt}A list of legal procedures associated with the decision, separated by a semi-colon.
\item[\code{is\_hearing}] \code{dummy}\hspace{5pt}A dummy variable indicating whether there was an oral hearing.
\item[\code{hearing\_date}] \code{date}\hspace{5pt}The date of the hearing in the format \code{YYYY-MM-DD}, if applicable. Coded \code{NA} if not applicable or if missing.
\item[\code{chamber}] \code{string}\hspace{5pt}The chamber of the Court that heard the case.
\item[\code{chamber\_size}] \code{numeric}\hspace{5pt}The size of the chamber that heard the case.
\item[\code{president\_id}] \code{numeric}\hspace{5pt}An ID number that uniquely identifies the judge who served as president in the proceeding.
\item[\code{president}] \code{string}\hspace{5pt}The name of the president. Coded \code{NA} if not applicable or if missing.
\item[\code{rapporteur\_id}] \code{numeric}\hspace{5pt}An ID number that uniquely identifies the judge-rapporteur in the proceeding. Coded \code{NA} if not applicable or if missing.
\item[\code{rapporteur}] \code{string}\hspace{5pt}The name of the judge-rapporteur in the proceeding.
\item[\code{advocate\_general\_id}] \code{numeric}\hspace{5pt}An ID number that uniquely identifies the advocate general assigned to the case, if applicable. Coded \code{NA} if not applicable or if missing.
\item[\code{advocate\_general}] \code{string}\hspace{5pt}The name of the advocate general assigned to the case, if applicable. Coded \code{NA} if not applicable or if missing.
\item[\code{subject\_matter\_category}] \code{string}\hspace{5pt}The subject matter tags assigned to the decision, separated by a semicolon. Coded \code{NA} if missing.
\item[\code{subject\_matter\_subcategory}] \code{string}\hspace{5pt}The subject matter tags subcategory assigned to the decision, separated by a semicolon. Coded \code{NA} if missing.
\end{description}
%--------------------------------------------------%
% dataset
%--------------------------------------------------%

\headerpage{procedures}{Legal procedures in CJEU decisions}{35}{12}

\subheading{Description}

This dataset includes information on legal procedures referenced in CJEU decisions. There is one observation per legal procedure per decision. There are many legal procedures, but the major ones are references for preliminary rulings, actions for failure to fulfill obligations, actions for annulment, actions for failure to act, damages for non-contractual liability, and appeals. The dataset indicates the ruling of the Court with respect to each legal procedure. For direct actions (all legal procedures except references for a preliminary ruling), the Court can rule that the plaintiff's application was successful, unfounded, or inadmissible. It can also issue an interlocutory ruling or dismiss the application.

\subheading{Variables}

\begin{description}[labelwidth=130pt, leftmargin=\dimexpr\labelwidth+\labelsep\relax, font=\normalfont, itemsep=10pt]
\item[\code{key\_id}] \code{numeric}\hspace{5pt}An ID number that uniquely identifies each observation in the dataset. 
\item[\code{court\_id}] \code{numeric}\hspace{5pt}An ID number that uniquely identifies each court. Coded \code{1} for \code{Court of Justice}, \code{2} for \code{General Court}, and \code{3} for \code{Civil Service Tribunal}.
\item[\code{court}] \code{string}\hspace{5pt}The name of the court. Either \code{Court of Justice}, \code{General Court}, or \code{Civil Service Tribunal}.
\item[\code{iuropa\_proceeding\_id}] \code{string}\hspace{5pt}An ID number that uniquely identifies each proceeding in the format \code{CJEU:X:\#\#\#\#:\#\#\#\#}, where \code{X} indicates the court (\code{C} for the Court of Justice, \code{T} for the General Court, and \code{F} for the Civil Service Tribunal), the first set of digits is the year the proceeding was opened, and the second set of digits is the number of the proceeding with leading zeros.
\item[\code{proceeding}] \code{string}\hspace{5pt}The proceeding number in the format \code{C-\#\#\#/\#\#} for Court of Justice proceedings, \code{T-\#\#\#/\#\#} for General Court proceedings, and \code{F-\#\#\#/\#\#} for Civil Service Tribunal proceedings. The first set of digits is the number of the proceeding and the second set of digits is the last two digits of the year the proceeding was opened. When multiple cases are joined, the proceeding number for the joined cases is the lowest case number.
\item[\code{proceeding\_year}] \code{numeric}\hspace{5pt}The year of the proceeding, used in \code{proceeding} and \code{proceeding\_id}. The year of the earliest case associated with the proceeded was opened.
\item[\code{proceeding\_number}] \code{numeric}\hspace{5pt}The number of the proceeding, used in \code{proceeding} and \code{proceeding\_id}. The number is assigned based on the order in which the earliest case associated with the proceeding was opened. Case numbers reset at the start of each calendar year.
\item[\code{proceeding\_date}] \code{date}\hspace{5pt}The date the proceeding was opened in the format \code{YYYY-MM-DD}. If multiple cases were joined together, this is the date for the earliest of the joined cases.
\item[\code{iuropa\_decision\_id}] \code{string}\hspace{5pt}An ID number that uniquely identifies each decision in the format format \code{CJEU:X:\#\#\#\#:\#\#\#\#:Y:\#\#\#\#\#\#\#\#}, where \code{X} is a letter that indicates the court in question (\code{C} for the Court of Justice, \code{T} for the General Court, and \code{F} for the Civil Service Tribunal, consistent with the letters used in ECLI numbers), the first 4-digit number is the year of the case, the second 4-digit number is the case number (preceded by zeros), \code{Y} is a letter that indicates the type of the document, and the last 8-digit number is the date of the document (in the format \code{YYYYMMDD}). Possible values for \code{Y} are based on the document type indicators in CELEX numbers: \code{J} for judgments, \code{O} for orders, \code{V} for opinions of the Court, \code{D} for decisions, \code{X} for rulings, \code{S} for seizure orders, \code{T} for third-party proceedings, \code{C} for AG opinions, and \code{P} for AG views.
\item[\code{ecli}] \code{string}\hspace{5pt}The ECLI number of the decision. ECLI numbers are assigned by the Court. ECLI numbers have the format \code{ECLI:EU:X:\#\#\#\#:\#\#\#}, where \code{X} is a letter that indicates the court that issued the decision (\code{C} for the Court of Justice, \code{T} for the General Court, and \code{F} for the Civil Service Tribunal), the first set of digits is the year the document was published, and the second set of digits is the number of the decision within the year. Note that ECLI numbers are used for documents besides decisions, so they may not be consecutive for decisions. Note also that the year used in ECLI numbers (the year of publication) is often different from the date in CELEX numbers (the year of the earliest case associated with the proceeding) for the same document.
\item[\code{celex}] \code{string}\hspace{5pt}The CELEX number of the document if the document appears in EUR-Lex. CELEX numbers for CJEU documents have the format \code{6\#\#\#\#XX\#\#\#\#}, where the first set of digits is the year of the proceeding, \code{XX} is a two-letter code that indicates the type of the document, and the last set of digits is the number of the proceeding. If there are multiple CELEX numbers for the same type of document for the same proceeding, subsequent documents will include a suffix in the format \code{(\#\#)}, where the digits indicate the number of the document.
\item[\code{decision\_date}] \code{date}\hspace{5pt}The date that the decision was delivered in the format \code{YYYY-MM-DD}. Coded \code{NA} if missing.
\item[\code{decision\_type\_id}] \code{numeric}\hspace{5pt}An ID number that uniquely identifies each type of decision (see \code{decision\_type}). Coded \code{1} for \code{Judgment}, \code{2} for \code{Order}, \code{3} for \code{Opinion}, \code{4} for \code{Decision}, \code{5} for \code{Ruling}, and \code{6} for \code{AG opinion}.
\item[\code{decision\_type}] \code{string}\hspace{5pt}The type of the decision. Coded \code{NA} if missing. Possible values include: \code{Judgment}, \code{Order}, \code{Opinion}, \code{Decision}, \code{Ruling}, and \code{AG opinion}.
\item[\code{procedure\_type\_id}] \code{numeric}\hspace{5pt}An ID number that uniquely identifies each type of legal procedure.
\item[\code{procedure\_type}] \code{string}\hspace{5pt}The type of the legal procedure.
\item[\code{successful}] \code{dummy}\hspace{5pt}A dummy variable indicating whether the Court ruled that the applicant was successful.
\item[\code{unfounded}] \code{dummy}\hspace{5pt}A dummy variable indicating whether the Court ruled that the applicant’s claim was unfounded. 
\item[\code{inadmissible}] \code{dummy}\hspace{5pt}A dummy variable indicating whether the applicant’s claim was inadmissible. 
\item[\code{interlocutory}] \code{dummy}\hspace{5pt}A dummy variable indicating whether the Court issued an interlocutory decision.
\item[\code{dismissed}] \code{dummy}\hspace{5pt}A dummy variable indicating whether the court dismissed the application’s claim.
\end{description}
%--------------------------------------------------%
% dataset
%--------------------------------------------------%

\headerpage{assignments}{Chamber composition for CJEU decisions}{35}{12}

\subheading{Description}

This dataset includes information on which judges were assigned to the chamber that issued each decision. There is one observation per judge per decision. The dataset also indicates which judge served as the president (or acting president) of the chamber and which judge served as the judge-rapporteur (the judge who manages the proceeding and who drafts the opinion).

\subheading{Variables}

\begin{description}[labelwidth=130pt, leftmargin=\dimexpr\labelwidth+\labelsep\relax, font=\normalfont, itemsep=10pt]
\item[\code{key\_id}] \code{numeric}\hspace{5pt}An ID number that uniquely identifies each observation in the dataset. 
\item[\code{court\_id}] \code{numeric}\hspace{5pt}An ID number that uniquely identifies each court. Coded \code{1} for \code{Court of Justice}, \code{2} for \code{General Court}, and \code{3} for \code{Civil Service Tribunal}.
\item[\code{court}] \code{string}\hspace{5pt}The name of the court. Either \code{Court of Justice}, \code{General Court}, or \code{Civil Service Tribunal}.
\item[\code{iuropa\_proceeding\_id}] \code{string}\hspace{5pt}An ID number that uniquely identifies each proceeding in the format \code{CJEU:X:\#\#\#\#:\#\#\#\#}, where \code{X} indicates the court (\code{C} for the Court of Justice, \code{T} for the General Court, and \code{F} for the Civil Service Tribunal), the first set of digits is the year the proceeding was opened, and the second set of digits is the number of the proceeding with leading zeros.
\item[\code{proceeding}] \code{string}\hspace{5pt}The proceeding number in the format \code{C-\#\#\#/\#\#} for Court of Justice proceedings, \code{T-\#\#\#/\#\#} for General Court proceedings, and \code{F-\#\#\#/\#\#} for Civil Service Tribunal proceedings. The first set of digits is the number of the proceeding and the second set of digits is the last two digits of the year the proceeding was opened. When multiple cases are joined, the proceeding number for the joined cases is the lowest case number.
\item[\code{proceeding\_year}] \code{numeric}\hspace{5pt}The year of the proceeding, used in \code{proceeding} and \code{proceeding\_id}. The year of the earliest case associated with the proceeded was opened.
\item[\code{proceeding\_number}] \code{numeric}\hspace{5pt}The number of the proceeding, used in \code{proceeding} and \code{proceeding\_id}. The number is assigned based on the order in which the earliest case associated with the proceeding was opened. Case numbers reset at the start of each calendar year.
\item[\code{proceeding\_date}] \code{date}\hspace{5pt}The date the proceeding was opened in the format \code{YYYY-MM-DD}. If multiple cases were joined together, this is the date for the earliest of the joined cases.
\item[\code{iuropa\_decision\_id}] \code{string}\hspace{5pt}An ID number that uniquely identifies each decision in the format format \code{CJEU:X:\#\#\#\#:\#\#\#\#:Y:\#\#\#\#\#\#\#\#}, where \code{X} is a letter that indicates the court in question (\code{C} for the Court of Justice, \code{T} for the General Court, and \code{F} for the Civil Service Tribunal, consistent with the letters used in ECLI numbers), the first 4-digit number is the year of the case, the second 4-digit number is the case number (preceded by zeros), \code{Y} is a letter that indicates the type of the document, and the last 8-digit number is the date of the document (in the format \code{YYYYMMDD}). Possible values for \code{Y} are based on the document type indicators in CELEX numbers: \code{J} for judgments, \code{O} for orders, \code{V} for opinions of the Court, \code{D} for decisions, \code{X} for rulings, \code{S} for seizure orders, \code{T} for third-party proceedings, \code{C} for AG opinions, and \code{P} for AG views.
\item[\code{ecli}] \code{string}\hspace{5pt}The ECLI number of the decision. ECLI numbers are assigned by the Court. ECLI numbers have the format \code{ECLI:EU:X:\#\#\#\#:\#\#\#}, where \code{X} is a letter that indicates the court that issued the decision (\code{C} for the Court of Justice, \code{T} for the General Court, and \code{F} for the Civil Service Tribunal), the first set of digits is the year the document was published, and the second set of digits is the number of the decision within the year. Note that ECLI numbers are used for documents besides decisions, so they may not be consecutive for decisions. Note also that the year used in ECLI numbers (the year of publication) is often different from the date in CELEX numbers (the year of the earliest case associated with the proceeding) for the same document.
\item[\code{celex}] \code{string}\hspace{5pt}The CELEX number of the document if the document appears in EUR-Lex. CELEX numbers for CJEU documents have the format \code{6\#\#\#\#XX\#\#\#\#}, where the first set of digits is the year of the proceeding, \code{XX} is a two-letter code that indicates the type of the document, and the last set of digits is the number of the proceeding. If there are multiple CELEX numbers for the same type of document for the same proceeding, subsequent documents will include a suffix in the format \code{(\#\#)}, where the digits indicate the number of the document.
\item[\code{decision\_date}] \code{date}\hspace{5pt}The date that the decision was delivered in the format \code{YYYY-MM-DD}. Coded \code{NA} if missing.
\item[\code{decision\_type\_id}] \code{numeric}\hspace{5pt}An ID number that uniquely identifies each type of decision (see \code{decision\_type}). Coded \code{1} for \code{Judgment}, \code{2} for \code{Order}, \code{3} for \code{Opinion}, \code{4} for \code{Decision}, \code{5} for \code{Ruling}, and \code{6} for \code{AG opinion}.
\item[\code{decision\_type}] \code{string}\hspace{5pt}The type of the decision. Coded \code{NA} if missing. Possible values include: \code{Judgment}, \code{Order}, \code{Opinion}, \code{Decision}, \code{Ruling}, and \code{AG opinion}.
\item[\code{judge\_id}] \code{numeric}\hspace{5pt}An ID number that uniquely identifies each individual.
\item[\code{last\_name}] \code{string}\hspace{5pt}The last name of the judge. 
\item[\code{president}] \code{dummy}\hspace{5pt}A dummy variable indicating whether the judge server as the president in the proceeding.
\item[\code{rapporteur}] \code{dummy}\hspace{5pt}A dummy variable indicating whether the judge served as the judge-rapporteur in the proceeding.
\item[\code{acting}] \code{dummy}\hspace{5pt}A dummy variable indicating whether the judge was serving in an acting role.
\end{description}
%--------------------------------------------------%
% dataset
%--------------------------------------------------%

\headerpage{submissions}{Observations and interventions in CJEU decisions}{35}{12}

\subheading{Description}

This dataset includes information on submissions by third parties referenced in CJEU decisions. There is one observation per submission per decision. Submissions include observations, oral observations, and interventions. The dataset indicates the type of the submission, type of actor responsible for the submission (member state, EU institution, company, individual, etc.), and the side (applicant or defendant) that the actor supported, if applicable.

\subheading{Variables}

\begin{description}[labelwidth=130pt, leftmargin=\dimexpr\labelwidth+\labelsep\relax, font=\normalfont, itemsep=10pt]
\item[\code{key\_id}] \code{numeric}\hspace{5pt}An ID number that uniquely identifies each observation in the dataset. 
\item[\code{court\_id}] \code{numeric}\hspace{5pt}An ID number that uniquely identifies each court. Coded \code{1} for \code{Court of Justice}, \code{2} for \code{General Court}, and \code{3} for \code{Civil Service Tribunal}.
\item[\code{court}] \code{string}\hspace{5pt}The name of the court. Either \code{Court of Justice}, \code{General Court}, or \code{Civil Service Tribunal}.
\item[\code{iuropa\_proceeding\_id}] \code{string}\hspace{5pt}An ID number that uniquely identifies each proceeding in the format \code{CJEU:X:\#\#\#\#:\#\#\#\#}, where \code{X} indicates the court (\code{C} for the Court of Justice, \code{T} for the General Court, and \code{F} for the Civil Service Tribunal), the first set of digits is the year the proceeding was opened, and the second set of digits is the number of the proceeding with leading zeros.
\item[\code{proceeding}] \code{string}\hspace{5pt}The proceeding number in the format \code{C-\#\#\#/\#\#} for Court of Justice proceedings, \code{T-\#\#\#/\#\#} for General Court proceedings, and \code{F-\#\#\#/\#\#} for Civil Service Tribunal proceedings. The first set of digits is the number of the proceeding and the second set of digits is the last two digits of the year the proceeding was opened. When multiple cases are joined, the proceeding number for the joined cases is the lowest case number.
\item[\code{proceeding\_year}] \code{numeric}\hspace{5pt}The year of the proceeding, used in \code{proceeding} and \code{proceeding\_id}. The year of the earliest case associated with the proceeded was opened.
\item[\code{proceeding\_number}] \code{numeric}\hspace{5pt}The number of the proceeding, used in \code{proceeding} and \code{proceeding\_id}. The number is assigned based on the order in which the earliest case associated with the proceeding was opened. Case numbers reset at the start of each calendar year.
\item[\code{proceeding\_date}] \code{date}\hspace{5pt}The date the proceeding was opened in the format \code{YYYY-MM-DD}. If multiple cases were joined together, this is the date for the earliest of the joined cases.
\item[\code{iuropa\_decision\_id}] \code{string}\hspace{5pt}An ID number that uniquely identifies each decision in the format format \code{CJEU:X:\#\#\#\#:\#\#\#\#:Y:\#\#\#\#\#\#\#\#}, where \code{X} is a letter that indicates the court in question (\code{C} for the Court of Justice, \code{T} for the General Court, and \code{F} for the Civil Service Tribunal, consistent with the letters used in ECLI numbers), the first 4-digit number is the year of the case, the second 4-digit number is the case number (preceded by zeros), \code{Y} is a letter that indicates the type of the document, and the last 8-digit number is the date of the document (in the format \code{YYYYMMDD}). Possible values for \code{Y} are based on the document type indicators in CELEX numbers: \code{J} for judgments, \code{O} for orders, \code{V} for opinions of the Court, \code{D} for decisions, \code{X} for rulings, \code{S} for seizure orders, \code{T} for third-party proceedings, \code{C} for AG opinions, and \code{P} for AG views.
\item[\code{ecli}] \code{string}\hspace{5pt}The ECLI number of the decision. ECLI numbers are assigned by the Court. ECLI numbers have the format \code{ECLI:EU:X:\#\#\#\#:\#\#\#}, where \code{X} is a letter that indicates the court that issued the decision (\code{C} for the Court of Justice, \code{T} for the General Court, and \code{F} for the Civil Service Tribunal), the first set of digits is the year the document was published, and the second set of digits is the number of the decision within the year. Note that ECLI numbers are used for documents besides decisions, so they may not be consecutive for decisions. Note also that the year used in ECLI numbers (the year of publication) is often different from the date in CELEX numbers (the year of the earliest case associated with the proceeding) for the same document.
\item[\code{celex}] \code{string}\hspace{5pt}The CELEX number of the document if the document appears in EUR-Lex. CELEX numbers for CJEU documents have the format \code{6\#\#\#\#XX\#\#\#\#}, where the first set of digits is the year of the proceeding, \code{XX} is a two-letter code that indicates the type of the document, and the last set of digits is the number of the proceeding. If there are multiple CELEX numbers for the same type of document for the same proceeding, subsequent documents will include a suffix in the format \code{(\#\#)}, where the digits indicate the number of the document.
\item[\code{decision\_date}] \code{date}\hspace{5pt}The date that the decision was delivered in the format \code{YYYY-MM-DD}. Coded \code{NA} if missing.
\item[\code{decision\_type\_id}] \code{numeric}\hspace{5pt}An ID number that uniquely identifies each type of decision (see \code{decision\_type}). Coded \code{1} for \code{Judgment}, \code{2} for \code{Order}, \code{3} for \code{Opinion}, \code{4} for \code{Decision}, \code{5} for \code{Ruling}, and \code{6} for \code{AG opinion}.
\item[\code{decision\_type}] \code{string}\hspace{5pt}The type of the decision. Coded \code{NA} if missing. Possible values include: \code{Judgment}, \code{Order}, \code{Opinion}, \code{Decision}, \code{Ruling}, and \code{AG opinion}.
\item[\code{actor\_role\_id}] \code{numeric}\hspace{5pt}An ID variable that uniquely identifies each actor role. Coded \code{1} for actors submitting observations, \code{2} for actors submitting oral observations, and \code{3} for intervening parties.
\item[\code{actor\_role}] \code{string}\hspace{5pt}The role of the actor. Possible values include: \code{Observation}, \code{Oral observation}, and \code{Intervening Party}.
\item[\code{actor}] \code{string}\hspace{5pt}The name of the actor who made the submission.
\item[\code{actor\_type\_id}] \code{numeric}\hspace{5pt}An ID variable that uniquely identifies the type of the actor. Coded \code{1} for \code{EU institution}, \code{2} for \code{Employee of EU institution}, \code{3} for \code{State}, \code{4} for \code{Region}, \code{5} for \code{State institution}, \code{6} for \code{Employee of state institution}, \code{7} for \code{Company}, and \code{8} for \code{Individual}.
\item[\code{actor\_type}] \code{string}\hspace{5pt}The type of the actor. Possible values include: \code{EU institution}, \code{Employee of EU institution}, \code{State}, \code{Region}, \code{State institution}, \code{Employee of state institution}, \code{Company}, and \code{Individual}. 
\item[\code{supporting\_id}] \code{numeric}\hspace{5pt}An ID variable that uniquely identifies the party that the actor is supporting in their submission, if applicable. Coded \code{1} for \code{Applicant}, \code{2} for \code{Defendant}, \code{3} for \code{Intervening party}, \code{4} for \code{Other party}, and \code{5} for \code{Third party}.
\item[\code{supporting}] \code{string}\hspace{5pt}The party that the actor is supporting in their submission, if applicable. Coded \code{NA} if not applicable. Possible values include: \code{Applicant}, \code{Defendant}, \code{Intervening party}, \code{Other party}, and \code{Third party}.
\end{description}
%--------------------------------------------------%
% dataset
%--------------------------------------------------%

\headerpage{citations}{Citations in CJEU decisions}{35}{12}

\subheading{Description}

This dataset includes information on citations in CJEU decisions. There is one observation per unique citation per decision. It indicates each document that a decision cites, which paragraphs the citation occurs in, and which paragraphs and articles of the cited document are referenced (if applicable). The cited documents are identified by their CELEX numbers.

\subheading{Variables}

\begin{description}[labelwidth=130pt, leftmargin=\dimexpr\labelwidth+\labelsep\relax, font=\normalfont, itemsep=10pt]
\item[\code{key\_id}] \code{numeric}\hspace{5pt}An ID number that uniquely identifies each observation in the dataset. 
\item[\code{court\_id}] \code{numeric}\hspace{5pt}An ID number that uniquely identifies each court. Coded \code{1} for \code{Court of Justice}, \code{2} for \code{General Court}, and \code{3} for \code{Civil Service Tribunal}.
\item[\code{court}] \code{string}\hspace{5pt}The name of the court. Either \code{Court of Justice}, \code{General Court}, or \code{Civil Service Tribunal}.
\item[\code{iuropa\_proceeding\_id}] \code{string}\hspace{5pt}An ID number that uniquely identifies each proceeding in the format \code{CJEU:X:\#\#\#\#:\#\#\#\#}, where \code{X} indicates the court (\code{C} for the Court of Justice, \code{T} for the General Court, and \code{F} for the Civil Service Tribunal), the first set of digits is the year the proceeding was opened, and the second set of digits is the number of the proceeding with leading zeros.
\item[\code{proceeding}] \code{string}\hspace{5pt}The proceeding number in the format \code{C-\#\#\#/\#\#} for Court of Justice proceedings, \code{T-\#\#\#/\#\#} for General Court proceedings, and \code{F-\#\#\#/\#\#} for Civil Service Tribunal proceedings. The first set of digits is the number of the proceeding and the second set of digits is the last two digits of the year the proceeding was opened. When multiple cases are joined, the proceeding number for the joined cases is the lowest case number.
\item[\code{proceeding\_year}] \code{numeric}\hspace{5pt}The year of the proceeding, used in \code{proceeding} and \code{proceeding\_id}. The year of the earliest case associated with the proceeded was opened.
\item[\code{proceeding\_number}] \code{numeric}\hspace{5pt}The number of the proceeding, used in \code{proceeding} and \code{proceeding\_id}. The number is assigned based on the order in which the earliest case associated with the proceeding was opened. Case numbers reset at the start of each calendar year.
\item[\code{proceeding\_date}] \code{date}\hspace{5pt}The date the proceeding was opened in the format \code{YYYY-MM-DD}. If multiple cases were joined together, this is the date for the earliest of the joined cases.
\item[\code{iuropa\_decision\_id}] \code{string}\hspace{5pt}An ID number that uniquely identifies each decision in the format format \code{CJEU:X:\#\#\#\#:\#\#\#\#:Y:\#\#\#\#\#\#\#\#}, where \code{X} is a letter that indicates the court in question (\code{C} for the Court of Justice, \code{T} for the General Court, and \code{F} for the Civil Service Tribunal, consistent with the letters used in ECLI numbers), the first 4-digit number is the year of the case, the second 4-digit number is the case number (preceded by zeros), \code{Y} is a letter that indicates the type of the document, and the last 8-digit number is the date of the document (in the format \code{YYYYMMDD}). Possible values for \code{Y} are based on the document type indicators in CELEX numbers: \code{J} for judgments, \code{O} for orders, \code{V} for opinions of the Court, \code{D} for decisions, \code{X} for rulings, \code{S} for seizure orders, \code{T} for third-party proceedings, \code{C} for AG opinions, and \code{P} for AG views.
\item[\code{ecli}] \code{string}\hspace{5pt}The ECLI number of the decision. ECLI numbers are assigned by the Court. ECLI numbers have the format \code{ECLI:EU:X:\#\#\#\#:\#\#\#}, where \code{X} is a letter that indicates the court that issued the decision (\code{C} for the Court of Justice, \code{T} for the General Court, and \code{F} for the Civil Service Tribunal), the first set of digits is the year the document was published, and the second set of digits is the number of the decision within the year. Note that ECLI numbers are used for documents besides decisions, so they may not be consecutive for decisions. Note also that the year used in ECLI numbers (the year of publication) is often different from the date in CELEX numbers (the year of the earliest case associated with the proceeding) for the same document.
\item[\code{celex}] \code{string}\hspace{5pt}The CELEX number of the document if the document appears in EUR-Lex. CELEX numbers for CJEU documents have the format \code{6\#\#\#\#XX\#\#\#\#}, where the first set of digits is the year of the proceeding, \code{XX} is a two-letter code that indicates the type of the document, and the last set of digits is the number of the proceeding. If there are multiple CELEX numbers for the same type of document for the same proceeding, subsequent documents will include a suffix in the format \code{(\#\#)}, where the digits indicate the number of the document.
\item[\code{decision\_date}] \code{date}\hspace{5pt}The date that the decision was delivered in the format \code{YYYY-MM-DD}. Coded \code{NA} if missing.
\item[\code{decision\_type\_id}] \code{numeric}\hspace{5pt}An ID number that uniquely identifies each type of decision (see \code{decision\_type}). Coded \code{1} for \code{Judgment}, \code{2} for \code{Order}, \code{3} for \code{Opinion}, \code{4} for \code{Decision}, \code{5} for \code{Ruling}, and \code{6} for \code{AG opinion}.
\item[\code{decision\_type}] \code{string}\hspace{5pt}The type of the decision. Coded \code{NA} if missing. Possible values include: \code{Judgment}, \code{Order}, \code{Opinion}, \code{Decision}, \code{Ruling}, and \code{AG opinion}.
\item[\code{cited\_ecli}] \code{string}\hspace{5pt}The ECLI number of the cited document, if applicable. Coded \code{NA} if not applicable or if missing.
\item[\code{cited\_celex}] \code{string}\hspace{5pt}The CELEX number of the cited document, if applicable. Coded \code{NA} if not applicable or if missing.
\item[\code{cited\_title}] \code{string}\hspace{5pt}The title of the cited document.
\item[\code{cited\_document\_type\_id}] \code{numeric}\hspace{5pt}An ID number that uniquely identifies each type of document.
\item[\code{cited\_document\_type}] \code{string}\hspace{5pt}The type of the cited document.
\item[\code{paragraphs}] \code{string}\hspace{5pt}The paragraphs in the citing document where the cited document is cited, separated by a comma. Coded \code{NA} if not applicable or if missing.
\item[\code{cited\_paragraphs}] \code{string}\hspace{5pt}The numbers of the paragraphs of the document that are cited, separated by a comma, if applicable. Coded \code{NA} if not applicable or if missing.
\item[\code{cited\_articles}] \code{string}\hspace{5pt}The numbers of the articles of the document that are cited, separated by a comma, if applicable. Coded \code{NA} if not applicable or if missing.
\end{description}
%--------------------------------------------------%
% dataset
%--------------------------------------------------%

\headerpage{judges}{CJEU judges and Advocates General}{35}{12}

\subheading{Description}

This dataset includes information on each judge and advocate-general who has served on CJEU. There is one observation per individual who has served as a judge or advocate-general. The dataset indicates each judge's member state, gender, appointment dates, and professional history, including whether each judge has prior experience as a judge, lawyer, politician, civil servant, or academic.

\subheading{Variables}

\begin{description}[labelwidth=130pt, leftmargin=\dimexpr\labelwidth+\labelsep\relax, font=\normalfont, itemsep=10pt]
\item[\code{key\_id}] \code{numeric}\hspace{5pt}An ID number that uniquely identifies each observation in the dataset. 
\item[\code{judge\_id}] \code{numeric}\hspace{5pt}An ID number that uniquely identifies each judge or Advocate General.
\item[\code{full\_name}] \code{string}\hspace{5pt}The full name of the judge.
\item[\code{first\_name}] \code{string}\hspace{5pt}The first name of the judge.
\item[\code{last\_name}] \code{string}\hspace{5pt}The last name of the judge.
\item[\code{last\_name\_latin}] \code{string}\hspace{5pt}The last name of the judge using only standard Latin characters without any diacritical marks (to avoid character encoding problems).
\item[\code{last\_name\_label}] \code{string}\hspace{5pt}The last name of the judge formatted as a label to use in plots. This label uniquely identifies judges with the same last name by including a first initial where necessary.
\item[\code{last\_name\_latin\_label}] \code{string}\hspace{5pt}The last name of the judge, using only standard Latin characters, formatted as a label to use in plots. This label uniquely identifies judges with the same last name by including a first initial where necessary.
\item[\code{member\_state\_id}] \code{numeric}\hspace{5pt}An ID number that uniquely identifies each member state. The ID numbers are assigned when the member states are sorted by date of accession, then alphabetically.
\item[\code{member\_state}] \code{string}\hspace{5pt}The member state that nominated the judge or advocate general. 
\item[\code{member\_state\_code}] \code{string}\hspace{5pt}A two character ID number assigned by the Commission that uniquely identifies each member state.
\item[\code{birth\_year}] \code{numeric}\hspace{5pt}The year the judge was born. 
\item[\code{female}] \code{dummy}\hspace{5pt}A dummy variable indicating whether the judge is female.
\item[\code{judge\_court\_of\_justice}] \code{numeric}\hspace{5pt}A dummy variable indicating whether the individual is or was a judge at the Court of Justice.
\item[\code{judge\_general\_court}] \code{numeric}\hspace{5pt}A dummy variable indicating whether the individual is or was a judge at the General Court.
\item[\code{judge\_civil\_service\_tribunal}] \code{numeric}\hspace{5pt}A dummy variable indicating whether the individual was a judge at the now-defunct Civil Service Tribunal.
\item[\code{advocate\_general}] \code{numeric}\hspace{5pt}A dummy variable indicating whether the individual is or was an advocate general at the Court of Justice.
\item[\code{start\_date}] \code{date}\hspace{5pt}The date that the individual started their first position at the Court of Justice of the European Union in the format \code{YYYY-MM-DD}.
\item[\code{end\_date}] \code{date}\hspace{5pt}The date that the individual left their last position at the Court of Justice of the European Union in the format \code{YYYY-MM-DD}. Coded \code{NA} if the individual still has a position at the Court. 
\item[\code{start\_date\_court\_of\_justice}] \code{date}\hspace{5pt}The date that the individual started their position as a judge at the Court of Justice, if applicable, in the format \code{YYYY-MM-DD}. Coded \code{NA} if not applicable.
\item[\code{end\_date\_court\_of\_justice}] \code{date}\hspace{5pt}The date that the individual left their position as a judge at the Court of Justice, if applicable, in the format \code{YYYY-MM-DD}. Coded \code{NA} if not applicable. Coded \code{NA} if the individual is currently still a judge.
\item[\code{start\_date\_general\_court}] \code{date}\hspace{5pt}The date that the individual started their position as a judge at the General Court, if applicable, in the format \code{YYYY-MM-DD}. Coded \code{NA} if not applicable.
\item[\code{end\_date\_general\_court}] \code{date}\hspace{5pt}The date that the individual left their position as a judge at the General Court, if applicable, in the format \code{YYYY-MM-DD}. Coded \code{NA} if not applicable. Coded \code{NA} if the individual is currently still a judge.
\item[\code{start\_date\_civil\_service\_tribunal}] \code{date}\hspace{5pt}The date that the individual started their position as a judge at the Civil Service Tribunal, if applicable, in the format \code{YYYY-MM-DD}. Coded \code{NA} if not applicable.
\item[\code{end\_date\_civil\_service\_tribunal}] \code{date}\hspace{5pt}The date that the individual left their position as a judge at the Civil Service Tribunal, if applicable, in the format \code{YYYY-MM-DD}. Coded \code{NA} if not applicable.
\item[\code{start\_date\_advocate\_general}] \code{date}\hspace{5pt}The date that the individual started their position as an adovcate general at the Court of Justice, if applicable, in the format \code{YYYY-MM-DD}. Coded \code{NA} if not applicable.
\item[\code{end\_date\_advocate\_general}] \code{date}\hspace{5pt}The date that the individual left their position as an advocate general at the Court of Justice, if applicable, in the format \code{YYYY-MM-DD}. Coded \code{NA} if not applicable. Coded \code{NA} if the individual is currently still an advocate general.
\item[\code{background\_judge}] \code{numeric}\hspace{5pt}A count of the number of positions that the judge or advocate general held as a judge before being nominated to the Court.
\item[\code{background\_lawyer}] \code{numeric}\hspace{5pt}A dummy variable indicating whether the judge or advocate general had a position as a lawyer before being nominated to the Court.
\item[\code{background\_politics}] \code{numeric}\hspace{5pt}A count of the number of positions that the judge or advocate general held as a politician before being nominated to the Court.
\item[\code{background\_civil\_servant}] \code{numeric}\hspace{5pt}A count of the number of positions that the judge or advocate general held as a politician before being nominated to the Court.
\item[\code{background\_academia}] \code{numeric}\hspace{5pt}A count of the number of positions that the judge or advocate general held in academia before being nominated to the Court.
\item[\code{university\_education}] \code{string}\hspace{5pt}A list of universities that the judge or advocate general studied at, separated by a semicolon. Coded \code{NA} if not applicable.
\item[\code{university\_work}] \code{string}\hspace{5pt}A list of universities that the judge or advocate general worked at, separated by a semicolon. Coded \code{NA} if not applicable.
\end{description}

%--------------------------------------------------%
% end document
%--------------------------------------------------%

\end{flushleft}

\end{document}
